\conclusions

\chapter*{CONCLUSIONES}
\addcontentsline{toc}{chapter}{CONCLUSIONES}
\markboth{CONCLUSIONES}{CONCLUSIONES}

A partir del desarrollo de la presente investigaci�n se logra arribar a las siguientes conclusiones generales:

\begin{enumerate}
	\item La toma de decisiones en contextos complejos con m�ltiples criterios constituye un proceso fundamental en diversos �mbitos profesionales, el cual se ve limitado por los m�todos manuales tradicionales debido a la subjetividad, la dificultad para manejar grandes vol�menes de datos y la incapacidad para procesar informaci�n imprecisa.
	
	\item El an�lisis de los m�todos MCDM permiti� establecer que herramientas como AHP, TOPSIS, ELECTRE y PROMETHEE, proporcionan un marco conceptual s�lido para abordar problemas complejos de decisi�n, facilitando la evaluaci�n de alternativas bajo m�ltiples criterios.
	
	\item El uso de Python como lenguaje de programaci�n, junto con Visual Studio Code y herramientas CASE como Visual Paradigm, permiti� desarrollar una soluci�n que cumple con los requisitos establecidos de portabilidad y eficiencia.
	
	\item La metodolog�a de desarrollo �gil XP demostr� ser efectiva para este proyecto, permitiendo la implementaci�n iterativa e incremental del software mediante 17 historias de usuario organizadas en 4 iteraciones, lo que facilit� la adaptaci�n a los cambios y garantiz� la calidad del producto final.
	
	\item La arquitectura de N capas y el patr�n MVC implementados proporcionaron una estructura robusta y mantenible para el software, permitiendo una clara separaci�n de responsabilidades y facilitando la extensibilidad del sistema para incorporar nuevos m�todos MCDM.
	
	\item Las pruebas de software realizadas demostraron que la implementaci�n satisface las necesidades del cliente, con cero no conformidades despu�s de cinco iteraciones de pruebas unitarias y de aceptaci�n.
	
	\item La validaci�n mediante Pruebas de Veracidad confirm� la exactitud del software con una concordancia del 100\% con los resultados publicados en literatura cient�fica, validando la correcta implementaci�n de los m�todos MCDM.
	
	\item La soluci�n desarrollada constituye una contribuci�n significativa al campo de la ingenier�a inform�tica, proporcionando una herramienta tecnol�gica que optimiza los procesos de toma de decisiones multicriterio, reduce costos y tiempos de an�lisis, y ofrece una plataforma flexible y escalable para futuras investigaciones.
\end{enumerate}