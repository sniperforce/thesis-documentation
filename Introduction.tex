\introduction
\chapter*{INTRODUCCIÓN}
\addcontentsline{toc}{chapter}{INTRODUCCIÓN}
\markboth{INTRODUCCIÓN}{INTRODUCCIÓN}

La toma de decisiones constituye un proceso cognitivo fundamental que permea múltiples dominios organizacionales, desde la administración de recursos humanos hasta la formulación de estrategias corporativas. Los procesos decisionales contemporáneos se caracterizan por su complejidad inherente, manifestada a través de la presencia de intereses divergentes, la participación de múltiples actores y la necesidad de evaluar factores de naturaleza heterogénea. Esta complejidad se intensifica exponencialmente cuando los decisores enfrentan escenarios que involucran volúmenes considerables de información, alternativas interdependientes y criterios de evaluación múltiples y frecuentemente conflictivos \citep{cinelli2021}. Como establecen Thomas et al. \citep{thomas2023}, los problemas de toma de decisiones complejos requieren la utilización de ecuaciones matemáticas, estadísticas multivariadas y dispositivos computacionales para calcular y estimar soluciones de manera automatizada, evidenciando las limitaciones de los enfoques tradicionales basados únicamente en el juicio humano.

La resolución de problemas decisionales multicriterio implica el procesamiento cognitivo de múltiples dimensiones de información y la integración de juicios de expertos, procesos que frecuentemente exceden las capacidades de procesamiento del sistema cognitivo humano. La literatura científica documenta consistentemente que el juicio intuitivo presenta limitaciones sistemáticas en contextos de alta complejidad, siendo susceptible a sesgos cognitivos, efectos de anclaje y errores de confirmación que comprometen la objetividad del análisis \citep{berthet2021}. \citep{fugener2021} demuestran que la toma de decisiones estratégicas se ve particularmente afectada por recursos cognitivos limitados, especialmente en escenarios caracterizados por alta incertidumbre y retroalimentación demorada. Investigaciones recientes evidencian que los sistemas de soporte automatizados superan consistentemente a los predictores humanos en tareas de juicio multivariado \citep{rahim2023}, estableciendo así la fundamentación científica para la implementación de Tecnologías de la Información y las Comunicaciones (TIC) como herramientas de apoyo decisional en contextos organizacionales complejos.

En las últimas décadas, los métodos de Toma de Decisiones Multicriterio \ac{MCDM} han experimentado un crecimiento exponencial en su adopción y desarrollo, consolidándose como herramientas analíticas fundamentales para abordar problemas decisionales de alta complejidad \citep{thomas2023}. Estos métodos proporcionan marcos metodológicos estructurados que permiten la descomposición sistemática de problemas multidimensionales, facilitando la evaluación simultánea de criterios heterogéneos —cuantitativos y cualitativos— y la identificación de alternativas óptimas dentro de espacios de decisión complejos \citep{cinelli2021}. La literatura científica documenta más de 10,000 publicaciones en el período 2012-2022, evidenciando su aplicabilidad transversal en dominios que abarcan desde la ingeniería y las finanzas hasta la gestión ambiental y la planificación estratégica organizacional \citep{thomas2023}. 

No obstante, la implementación manual de estos métodos presenta limitaciones sistemáticas bien documentadas que comprometen su efectividad práctica. Entre estas limitaciones se identifican: (i) la introducción de sesgos cognitivos inherentes al procesamiento humano que afectan la consistencia de las evaluaciones \citep{berthet2021}, (ii) la incapacidad para procesar eficientemente matrices de decisión de alta dimensionalidad que caracterizan problemas reales, (iii) las dificultades para mantener la coherencia en comparaciones pareadas cuando el número de criterios y alternativas se incrementa exponencialmente, y (iv) la ausencia de mecanismos automatizados para el tratamiento de información imprecisa, incompleta o caracterizada por incertidumbre epistémica. Estas limitaciones fundamentan la necesidad imperativa de desarrollar plataformas tecnológicas especializadas que automaticen los procesos computacionales subyacentes, preservando la rigurosidad metodológica mientras superan las restricciones cognitivas del análisis manual.

La revisión de la literatura científica acerca de la solución a problemas de toma de decisión con múltiples criterios y experto a permitido identificar los siguientes elementos:

\begin{itemize}
	\item Existencia de múltiples métodos para solucionar este tipo de problemas (ANP, AHP, PROMETHEE, TOPSIS, entre otros)
	\item No existen plataformas flexibles que permitan la comparación de resultados obtenido con diferentes métodos
	\item Muchas de las herramientas existentes son costosas, difíciles de usar o no permiten la integración de múltiples métodos MCDM.
	\item Los métodos manuales de toma de decisiones son propensos a errores, subjetividad y sesgos cognitivos.
\end{itemize}

Dada la \textbf{situación problemática}, se formula el siguiente \textbf{problema a resolver}:

\begin{quote}
	\textit{¿Cómo contribuir a la toma de decisiones usando múltiples criterios en diferentes ámbitos profesionales con la ayuda de las Tecnologías de la Información y la Comunicación?}
\end{quote}

El \textbf{objeto de estudio} serían los Métodos de Decisión Multicriterio y el \textbf{campo de acción} Software de apoyo a la toma de decisiones multicriterio.

El \textbf{objetivo general} de esta investigación es desarrollar un software de apoyo a la toma de decisiones basado en los métodos de toma de decisión multicriterio.

Para lograr este objetivo, se han definido los siguientes \textbf{objetivos específicos}:

\begin{enumerate}
	\item Caracterizar los métodos MCDM más utilizados en la toma de decisiones.
	\item Realizar el análisis de los software homólogos para obtener  las carencias existentes en la industria. 
	\item Elaborar una propuesta de Software que incluya la selección de las herramientas, marcos de trabajo, lenguaje de programación así como el análisis y diseño utilizando una metodología de desarrollo adecuada.
	\item Implementar la propuesta de Software a partir de los artefactos elaborados.
	\item Validar el software mediante casos de estudio reales o simulados, comparando los resultados con los reportados en la literatura científica.
\end{enumerate}

Para la realización de la problemática anterior se utilizaron los \textbf{métodos de investigación científica}, tanto teóricos como empíricos.

Los métodos teóricos utilizados son:

\begin{itemize}
	\item \textbf{Análisis histórico-lógico}: Este método permite estudiar la evolución de un fenómeno a través del tiempo, estableciendo relaciones entre sus diferentes etapas de desarrollo. Se utilizó para examinar la evolución de los métodos MCDM, identificando tendencias y patrones en su aplicación y desarrollo histórico.
	
	\item \textbf{Análisis-síntesis}: Proceso mental que consiste en descomponer un fenómeno en sus partes constitutivas para estudiarlas de forma aislada (análisis) y luego integrar los resultados para obtener una visión global (síntesis). Este método se aplicó para integrar y resumir la información relevante sobre los diferentes métodos MCDM, sus características y aplicaciones, permitiendo establecer relaciones entre conceptos y generar nuevos conocimientos.
	
	\item \textbf{Modelación}: Permite representar de forma simplificada fenómenos complejos mediante modelos que facilitan su estudio. Se utilizó para desarrollar el modelo matemático que sustenta el software MCDM propuesto, representando adecuadamente las relaciones entre alternativas y criterios.
\end{itemize}

Los métodos Empíricos utilizados son:

\begin{itemize}
	\item \textbf{Observación}: Consiste en la percepción directa y registro sistemático de fenómenos. Se aplicó durante las fases de prueba y validación del software, permitiendo recopilar datos sobre usabilidad y eficacia de la interfaz desde la perspectiva del usuario.
	
	\item \textbf{Revisión documental}: Abarca la recolección, selección y análisis de fuentes bibliográficas. Este método fue fundamental para construir el estado del arte y fundamentar teóricamente la investigación, proporcionando una base sólida para el diseño e implementación del software propuesto.
\end{itemize}

El documento se encuentra estructurado en introducción, tres capítulos, conclusiones, recomendaciones, referencias bibliográficas y los anexos.

El \textbf{primer capítulo} se titula ``Fundamentos y Referentes sobre los Software de Toma de Decisiones'' y tiene como objetivo principal analizar los elementos teóricos base de la investigación. Los conceptos esenciales relacionados con toma de decisión multicriterio. Además, posee una descripción de los métodos que resuelven problemas de toma de decisión, así como una breve caracterización de software existentes que dan solución al problema planteado. Se destacarán las herramientas, metodologías y lenguajes de programación que se van a utilizar en el desarrollo del sistema.

El \textbf{segundo capítulo} se titula ``Análisis y diseño de la propuesta de Software OptiChoice'', su objetivo principal es presentar la propuesta de software para la solución de problemas de toma de decisión con múltiples criterios y expertos. Este capítulo contiene los principales elementos que caracterizan la propuesta de solución. Se especifican las características y funcionalidades del sistema como los patrones de diseño y arquitectura a utilizar en el sistema entre otros elementos necesarios para concretar la propuesta, todo esto teniendo en cuenta los requisitos funcionales y no funcionales identificados.

El \textbf{tercer y último capítulo} titulado ``Implementación y Validación de la propuesta de Software OptiChoice'' consiste en presentar los resultados de la implementación y pruebas realizadas a la aplicación. Se describe el proceso de construcción de la plataforma software, así como las pruebas unitarias y de aceptación para validar su correcta elaboración y funcionamiento. Además de la realización de pruebas de veracidad a partir de artículos científicos para demostrar la confiabilidad del software.

